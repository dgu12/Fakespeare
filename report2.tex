\documentclass{article}
\usepackage{amsmath}
%\usepackage{subfigure}
\usepackage{subfig}
\usepackage{amsthm}
\usepackage{amssymb}
\usepackage{graphicx}
\usepackage{mdwlist}
\usepackage[colorlinks=true]{hyperref}
\usepackage{geometry}
\usepackage{titlesec}
\geometry{margin=1in}
\geometry{headheight=2in}
\geometry{top=2in}
\usepackage{palatino}
% \usepackage{mathrsfs}
\usepackage{fancyhdr}
\usepackage{paralist}
% \usepackage{todonotes}
\setlength{\marginparwidth}{2.15cm}
\usepackage{tikz}
\usetikzlibrary{positioning,shapes,backgrounds}
\usepackage{float} % Place figures where you ACTUALLY want it
\usepackage{comment}
\usepackage{ifthen}
\rhead{}
\lhead{}

\renewcommand{\baselinestretch}{1.15}

% Shortcuts for commonly used operators
\newcommand{\E}{\mathbb{E}}
\newcommand{\Var}{\operatorname{Var}}
\newcommand{\Cov}{\operatorname{Cov}}
\DeclareMathOperator{\argmin}{arg\,min}
\DeclareMathOperator{\argmax}{arg\,max}

% do not number subsection and below
\setcounter{secnumdepth}{1}

% custom format subsection
\titleformat*{\subsection}{\large\bfseries}

% set up the \question shortcut
\newcounter{question}[section]
\newenvironment{question}[1][]
  {\refstepcounter{question}\par\addvspace{1em}\textbf{Question~\Alph{question}\!
    \ifthenelse{\equal{#1}{}}{}{ [#1 points]}: }}
    {\par\vspace{\baselineskip}}

\newcounter{subquestion}[question]
\newenvironment{subquestion}[1][]
  {\refstepcounter{subquestion}\par\medskip\textbf{\roman{subquestion}.\!
    \ifthenelse{\equal{#1}{}}{}{ [#1 points]:}} }
  {\par\addvspace{\baselineskip}}

\titlespacing\section{0pt}{12pt plus 2pt minus 2pt}{0pt plus 2pt minus 2pt}
\titlespacing\subsection{0pt}{12pt plus 4pt minus 2pt}{0pt plus 2pt minus 2pt}
\titlespacing\subsubsection{0pt}{12pt plus 4pt minus 2pt}{0pt plus 2pt minus 2pt}


\head{%
  {\vbox{%
      \vspace{2mm}
      \large
      Machine Learning \& Data Mining \hfill
      Caltech CS/CNS/EE 155 \hfill \\[1pt]
      Miniproject 2\hfill
      March $13^{\text{rd}}$, 2016 \\
    }
  }
}

\begin{document}
\pagestyle{fancy}

\LARGE
\begin{center}
Miniproject 2: Generating Shakespearean Sonnets using First-Order Hidden Markov Models

\large
Daniel Gu \\
Victor Han \\
Eve Yuan
\end{center}

\normalsize
\medskip

\section{Overview}
In this project, we wanted to learn a generative model which can write Shakespearean sonnets. Using Shakespeare's 154 sonnets and Spenser's sonnets from his sonnet cycle \textit{Amoretti} as unlabeled training data, we tried to train a first-order hidden Markov model (HMM) to generate sonnets similar to ones Shakespeare could have written.
\par A lot of the challenge in this miniproject is that first-order HMMs seem to be too weak to be able to perform this learning task. Sonnets have structure at varying levels of granularity, such as the iambic pentameter, which means that the previous word affects the current word, the syllable count, in which all of the previous words in a line affect the current word, and the rhyme, which is a "long-distance" interaction among line endings. And this is just the \textit{syntactic} structure of the sonnet: the stanzas and couplet of a Shakespearean sonnet also need to be endowed with meaning. So a lot of the challenge was augmenting or clever tweaking our data to try to capture this structure using a first-order HMM.
\par More concretely, a challenge we ran into was the amount of time needed to train a first-order HMM. An off-the-self implementation such as the one by the Natural Language Toolkit takes a lot of time (a few seconds per iteration for 10 hidden states, significantly more at higher hidden states), and our implementation took similar (and usually longer) times to finish training the HMM. So it took a lot of time to check for errors in the EM algorithm implementation, and after it was debugged, a lot of time to look for a good number of hidden states for our HMM. We were also constrained in time since looking at higher numbers of hidden states took too long. [TODO: discuss other challenges here.]
\par Daniel wrote some parsing code as well as used an off-the-shelf implementation of unsupervised training for HMMs (nltk.tag.hmm) to look for the best number of hidden states, and wrote the code to create rhyming poems. [TODO: add what Eve and Victor did.]


\section{Data Manipulation}
Given text files of Shakespeare and Spenser's sonnets, we parsed them into individual poems, which was represented as a list of its words in order. We also did parsing of poems into sequences of syllables. [TODO: discuss more preprocessing]. This seemed best since lines are too coarse a level of granularity, and parsing words could allow our model to capture parts of the structure such as the meter. [TODO: other stuff?]

\section{Learning Algorithm}
The major algorithm we had to implement was the Baum-Welch algorithm for unsupervised training of first-order HMMs. The Baum-Welch algorithm is a variant of the expectation-maximization algorithm for HMMs. Given training data in the form of unlabeled sequences, we first randomly initialize our transition matrix $A$ and our observation matrix $O$, which together completely characterizes an HMM. We then run the expectation step, in which we calculate maximum-likelihood marginal probabilities using the forwards-backwards algorithm as a subroutine. In particular, the forwards algorithm calculates given $A$ and $O$ using a Viterbi-like dynamic programming algorithm the values
$$\alpha_Z(i) = \Pr[x^{1 : i}, y^i = Z | A, O]$$
that is, the probability of observing the length-$i$ prefix of a sequence $x$ with the $i$th hidden state $y^i$ being state $Z$. Similarly, the backwards algorithm is also a DP algorithm which calculates
$$\beta_Z^(i) = \Pr[x^{i : M}, y^i = Z | A, O]$$
that is, the probability of observing the length-$M - i$ suffix of sequence $x$ with the $i$th hidden state being $Z$.
Once we have these, we can calculate the marginal probabilities according to
$$\Pr[y^i = Z | x] = \frac{\alpha_Z(i)\beta_Z(i)}{\sum_{Z'}{\alpha_{Z'}(i)\beta_{Z'}(i)}}$$
$$\Pr[y^i = b, y^{i - 1} = a | x] = \frac{\alpha_a(i - 1)A_{a, b}O_{x^i, b}\beta_b(i)}{\sum_{a', b'}{\alpha_{a'}(i - 1)A_{a', b'}O_{x^i, b'}\beta_{b'}(i)}}$$
Once we have these marginal probabilities, we can then perform a maximization step, in which we update $A$ and $O$ according to these marginals via
$$A_{a, b} = \frac{\sum_{j = 1}^{N}{\sum_{i = 0}^{M_j}{\Pr[y_j^i = b, y_j^{i + 1} = a]}}}{\sum_{j = 1}^{N}{\sum_{i = 0}^{M_j}{\Pr[y_j^i = b]}}}$$
$$O_{w, z} = \frac{\sum_{j = 1}^{N}{\sum_{i = 0}^{M_j}{1_{[x_j^i = w]} \cdot \Pr[y_j^i = z]}}}{\sum_{j = 1}^{N}{\sum_{i = 0}^{M_j}{\Pr[y_j^i = z]}}}$$
We can then repeat this until it converges. For our implementation, we stopped when the change in log probability does not change significantly. [TODO: discuss numerical stability issues and other stuff].
\par We can then generate a sequence in accordance with our trained HMM by randomly choosing a start state (according to the distribution described by the HMM), and then randomly choosing state transitions and token emissions in accordance with the probabilities described by the HMM. To enforce the 10-syllable per line structure, we used a tool to count the number of syllables and stop until we had 10 (although, due to the limitations of the tool, some lines may end up with 11 syllables). [TODO: discuss more.]
\par To create rhyming poems, we first used Shakespeare's and Spenser's sonnets to create a "rhyming dictionary", which internally looks like a list of lists, with each inner list represents a rhyme equivalence class; that is, every word in the list rhymes with every other word in the list. We then seeded the end of each line of our sonnet with matching pairs according to the rhyme scheme (Shakespeare's more relaxed $abab$ $cdcd$ $efef$ $gg$ scheme), and then generated each line backwards in accordance with out trained HMM.

\section{Model Selection}
We didn't do anything too sophisticated to do model selection; we trained our HMMs with different numbers of hidden states, and then examined the poems and chose the number of hidden states which made the poems we thought were qualitatively best. Due to the time required to do unsupervised HMM training with large numbers of hidden states, we limited our search to numbers of hidden states below 17. [TODO: verify and discuss more if necessary].

\section{Conclusion}
[TODO: discuss poem quality and stuff]


\end{document}
