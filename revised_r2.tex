\documentclass{article}
\usepackage{amsmath}
%\usepackage{subfigure}
\usepackage{subfig}
\usepackage{amsthm}
\usepackage{amssymb}
\usepackage{graphicx}
\usepackage{mdwlist}
\usepackage[colorlinks=true]{hyperref}
\usepackage{geometry}
\usepackage{titlesec}
\geometry{margin=1in}
\geometry{headheight=2in}
\geometry{top=2in}
\usepackage{palatino}
% \usepackage{mathrsfs}
\usepackage{fancyhdr}
\usepackage{paralist}
% \usepackage{todonotes}
\setlength{\marginparwidth}{2.15cm}
\usepackage{tikz}
\usetikzlibrary{positioning,shapes,backgrounds}
\usepackage{float} % Place figures where you ACTUALLY want it
\usepackage{comment}
\usepackage{ifthen}
\rhead{}
\lhead{}

\renewcommand{\baselinestretch}{1.15}

% Shortcuts for commonly used operators
\newcommand{\E}{\mathbb{E}}
\newcommand{\Var}{\operatorname{Var}}
\newcommand{\Cov}{\operatorname{Cov}}
\DeclareMathOperator{\argmin}{arg\,min}
\DeclareMathOperator{\argmax}{arg\,max}

% do not number subsection and below
\setcounter{secnumdepth}{1}

% custom format subsection
\titleformat*{\subsection}{\large\bfseries}

% set up the \question shortcut
\newcounter{question}[section]
\newenvironment{question}[1][]
  {\refstepcounter{question}\par\addvspace{1em}\textbf{Question~\Alph{question}\!
    \ifthenelse{\equal{#1}{}}{}{ [#1 points]}: }}
    {\par\vspace{\baselineskip}}

\newcounter{subquestion}[question]
\newenvironment{subquestion}[1][]
  {\refstepcounter{subquestion}\par\medskip\textbf{\roman{subquestion}.\!
    \ifthenelse{\equal{#1}{}}{}{ [#1 points]:}} }
  {\par\addvspace{\baselineskip}}

\titlespacing\section{0pt}{12pt plus 2pt minus 2pt}{0pt plus 2pt minus 2pt}
\titlespacing\subsection{0pt}{12pt plus 4pt minus 2pt}{0pt plus 2pt minus 2pt}
\titlespacing\subsubsection{0pt}{12pt plus 4pt minus 2pt}{0pt plus 2pt minus 2pt}


\chead{%
  {\vbox{%
      \vspace{2mm}
      \large
      Machine Learning \& Data Mining \hfill
      Caltech CS/CNS/EE 155 \hfill \\[1pt]
      Miniproject 2\hfill
      March $13^{\text{rd}}$, 2016 \\
    }
  }
}

\begin{document}
\pagestyle{fancy}

\LARGE
\begin{center}
Miniproject 2: Generating Shakespearean Sonnets using First-Order Hidden Markov Models

\large
Daniel Gu \\
Victor Han \\
Eve Yuan
\end{center}

\normalsize
\medskip

\section{Overview}
In this project, we wanted to learn a generative model which can write Shakespearean sonnets. Using Shakespeare's 154 sonnets and Spenser's sonnets from his sonnet cycle \textit{Amoretti} as unlabeled training data, we tried to train a first-order hidden Markov model (HMM) to generate sonnets similar to ones Shakespeare could have written.
\par A lot of the challenge in this miniproject is that first-order HMMs seem to be too weak to be able to perform this learning task. Sonnets have structure at varying levels of granularity, such as the iambic pentameter, which means that the previous word affects the current word, the syllable count, in which all of the previous words in a line affect the current word, and the rhyme, which is a "long-distance" interaction among line endings. And this is just the \textit{syntactic} structure of the sonnet: the stanzas and couplet of a Shakespearean sonnet also need to be endowed with meaning. So a lot of the challenge was augmenting or clever tweaking our data to try to capture this structure using a first-order HMM.
\par More concretely, a challenge we ran into was the amount of time needed to train a first-order HMM. An off-the-self implementation such as the one by the Natural Language Toolkit takes a lot of time (a few seconds per iteration for 10 hidden states, significantly more at higher hidden states), and our implementation took similar (and usually longer) times to finish training the HMM. So it took a lot of time to check for errors in the EM algorithm implementation, and after it was debugged, a lot of time to look for a good number of hidden states for our HMM. We were also constrained in time since looking at higher numbers of hidden states took too long. To help alleviate the problem of too much training time, we tried training on subsets of the poems at a time, which did help in the debugging process. However, even with using only a few poems, using large numbers of hidden states still took their toll. Another problem appeared to be that Shakespeare does not repeat words very often. Thus, it seemed like we did not have very much training data on most individual words, and thus their roles were probably difficult for the HMM to figure out. A final problem was that it was difficult to judge the quality of the poems generated. So, what works and what doesn't work was difficult to determine.
\par Daniel wrote some parsing code as well as used an off-the-shelf implementation of unsupervised training for HMMs (nltk.tag.hmm) to look for the best number of hidden states, and wrote the code to create rhyming poems. Victor wrote most of the unsupervised training code, wrote a little preprocessing code, and also looked for the best number of hidden states. Eve wrote some parsing code, most of the visualization code, and most of the poem generation code. 


\section{Data Manipulation/Preprocessing}
Given text files of Shakespeare and Spenser's sonnets, we parsed them into individual poems, which was represented as a list of its words in order. This seemed best since lines or bigger are too coarse a level of granularity (we don't want to just recite resorted Shakespeare lines), using syllables would cause us to generate imagined words, and parsing words could allow our model to capture parts of the structure such as the meter. We did end up writing a syllable parser anyways, but using it was soon abandoned.
\par In tokenizing as words, we also made a few other decisions. We kept hyphenated words together as a single token. After all, there must be a reason they are connected, and keeping them together may make our poem more meaningful/consistent. We kept punctuation with the words they come after. We wanted to keep these kinds of voice in the poem, and figuring out where to place them in poem generation would be difficult. Parentheses, however, were removed in preprocessing because the HMM cannot model opening and closing parentheses by itself. Initially, we did not take out parentheses and soon discovered that we got random closing parentheses and opening parentheses with no clear place to put closing counterparts when generating poems.
\par Other than straight up processing and tokenizing into words, we also did processing based on the number of unique words a poem had. In the function, parseTokLimMin() in parse.py, given a number of poems to search through from Shakespeare and Spenser and a total number n of poems to find, we tokenize  only the n poems whose combination yields the fewest number of unique tokens. That is, if a combination of three particular poems uses 176 unique tokens in total, and all other combinations of three poems uses 177+ unique tokens, we choose the set of three poems with 176 unique tokens. The reason for this preprocessing is two-fold. The first is that if we have fewer unique tokens, then we have more data on each individual token, so that the HMM can better train on it. The second reason is that we are very constrained on training time, and having fewer tokens in general reduces the time needed to train.

\section{Learning Algorithm}
The major algorithm we had to implement was the Baum-Welch algorithm for unsupervised training of first-order HMMs. The Baum-Welch algorithm is a variant of the expectation-maximization algorithm for HMMs. Given training data in the form of unlabeled sequences, we first randomly initialize our transition matrix $A$ and our observation matrix $O$, which together completely characterizes an HMM. We then run the expectation step, in which we calculate maximum-likelihood marginal probabilities using the forwards-backwards algorithm as a subroutine. In particular, the forwards algorithm calculates given $A$ and $O$ using a Viterbi-like dynamic programming algorithm the values
$$\alpha_Z(i) = \Pr[x^{1 : i}, y^i = Z | A, O]$$
that is, the probability of observing the length-$i$ prefix of a sequence $x$ with the $i$th hidden state $y^i$ being state $Z$. Similarly, the backwards algorithm is also a DP algorithm which calculates
$$\beta_Z^(i) = \Pr[x^{i : M}, y^i = Z | A, O]$$
that is, the probability of observing the length-$M - i$ suffix of sequence $x$ with the $i$th hidden state being $Z$.
Once we have these, we can calculate the marginal probabilities according to
$$\Pr[y^i = Z | x] = \frac{\alpha_Z(i)\beta_Z(i)}{\sum_{Z'}{\alpha_{Z'}(i)\beta_{Z'}(i)}}$$
$$\Pr[y^i = b, y^{i - 1} = a | x] = \frac{\alpha_a(i - 1)A_{a, b}O_{x^i, b}\beta_b(i)}{\sum_{a', b'}{\alpha_{a'}(i - 1)A_{a', b'}O_{x^i, b'}\beta_{b'}(i)}}$$
Once we have these marginal probabilities, we can then perform a maximization step, in which we update $A$ and $O$ according to these marginals via
$$A_{a, b} = \frac{\sum_{j = 1}^{N}{\sum_{i = 0}^{M_j}{\Pr[y_j^i = b, y_j^{i + 1} = a]}}}{\sum_{j = 1}^{N}{\sum_{i = 0}^{M_j}{\Pr[y_j^i = b]}}}$$
$$O_{w, z} = \frac{\sum_{j = 1}^{N}{\sum_{i = 0}^{M_j}{1_{[x_j^i = w]} \cdot \Pr[y_j^i = z]}}}{\sum_{j = 1}^{N}{\sum_{i = 0}^{M_j}{\Pr[y_j^i = z]}}}$$
We can then repeat this until it converges. For our implementation, we stopped when the sum of the Frobenius norms of the difference between the old and new A matrices and of the difference between the old and new O matrices was very small compared to the first algorithm iteration difference.

\section{Naive Poem Generation}
\par Using the trained A and O matrices, we can then generate a sequence in accordance with our trained HMM by randomly choosing a start state (according to the distribution described by the HMM), and then randomly choosing state transitions and token emissions in accordance with the probabilities described by the HMM. To enforce the 10-syllable per line structure, we used a tool to count the number of syllables and stop until we had 10 (although, due to the limitations of the tool, some lines may end up with 11 syllables). 
\par We trained our naive HMM model using a variety of subsets of poems and number of hidden states. Models trained with a high number of hidden states and small number of poems naturally had good performance, since there is a high probability of matching Shakespeare's words or phrases:

\setlength{\leftskip}{3cm}

\noindent\textit{
When thou o present with I not think too, whom \\
Me, now I do, doing thee vantage, do \\
Do I not my self I in still with sighs \\
Them, moan? What merit do I not thy mind, those \\
Those that whom that thou my best thy part I \\
I to whom bending that against my self with \\
With thee vantage, double-vantage me. \\
Me. Such is my self, to my self me, art \\
Art present blind. With sighs and they with thee. Glory: \\
Or heart am still with whom time thou shalt on \\
On me do I not my self with thee vantage, \\
Partake? Double-vantage me. Such is \\
Is my love, to my self I do, doing \\
Doing thee partake? Do I not my self
}

\setlength{\leftskip}{0pt}

This poem was generated on the 3 Shakespeare sonnets with the least unique tokens with the number of hidden states equal to the number of unique tokens (176). It still doesn't make semantic sense, but a lot of phrases are natural English phrases.
\par We also did tests with all of the poems on a small number of hidden states. 12 hidden states seemed to generate good poems, compared to the number of hidden states around it:

\setlength{\leftskip}{3cm}

\noindent\textit{
Have sings hath thou heretic, they best all love, is \\
Such nought the old, all the for when name have, \\
On thy way, with admitted muse with wiry \\
The burthen in it know more though alone the \\
They like among the when after new worth then \\
So common thy quietus take, them if steal \\
Do I th' which day, within therefore desire \\
Shall may not to subjects why all and of \\
From repair it may nor faith be came sweets \\
Divine, summer's (my blind. Cross. Maturity, \\
Friends in they injurious hardest the \\
Believe hymns blooms nor I her longer hast cupid \\
Deceived. Strange: let thy sick wane and all my \\
Within, dear o I yet eyes for what compare
}

\setlength{\leftskip}{0pt}

Although the model is clearly learning something (it seems like it understands conjunctions to some degree, for example) the bulk of the poem is all totally nonsense. This model was trained only on Shakespeare's poems. If we train with Spenser's poem as well, we get a poem like

\setlength{\leftskip}{3cm}

\noindent\textit{
Excuse and pride, insight, is (being weave. \\
They with too fair it to ye of one deny, \\
Breaches is wrack. Nor at make treading been to \\
Down hell bring for thy runs my heart: to base \\
True garden come mind: and countenance habitation \\
With in new beauty, to what assurance: \\
Sake: was that not so crew: where with angel's \\
Left pine, to some the wander only do my \\
Not your the thou bend, heresy minutes last their \\
Alone, shinedst whom it is my (my laugh by not, \\
And would and tread death's a fair from feeble \\
Sickness pine will the blossom of it eyes \\
Leach turns contrained do they weave. And alone, they \\
Skill: this to absence on spotless defeat,
}

\setlength{\leftskip}{0pt}

The poems generated are not clearly better than those generated with Shakespeare's sonnets alone. It is possible to discern a shift in style, though; the model trained only on Shakespeare sounds (to the authors, at least) more like Shakespeare than ones trained on both Shakespeare's and Spenser's sonnets, in which a bit of Spenser's style can be seen. Also, note that the model is not particularly good at replicating the meters.

\par Due to the time needed to train unsupervised HMMs, exploration on other types of models was not possible.


\section{Better Poem Generation}
\par To create rhyming poems, we first used Shakespeare's and Spenser's sonnets to create a "rhyming dictionary", which internally looks like a list of lists, with each inner list represents a rhyme equivalence class; that is, every word in the list rhymes with every other word in the list. We then seeded the end of each line of our sonnet with matching pairs (if our HMM training set included these tokens) according to the rhyme scheme (Shakespeare's more relaxed $abab$ $cdcd$ $efef$ $gg$ scheme), and then generated each line backwards in accordance with our trained HMM.
\par An example of a rhyming poem is

\setlength{\leftskip}{3cm}

\noindent\textit{
She time's her long with old body so elsewhere, \\
Suit painted nor me as so a to foregone, \\
Name do cherish: likewise whose her your near. \\
Purchase love's of lusty predict as her worth. Gone. \\
Heavy me and what deep to you taken. Poor. \\
With greater by thy outward heavenly to glad, \\
Past him her another eternal being store. \\
Hath o banks, of make like loves, to me sad. \\
Splendour golden bower in my tongue, flatter read. \\
With for mutual done confounds self he canopy, \\
Of strength them herself her so the basest plead: \\
Hold, to both sang. By before? Eternity, \\
Night seen, for stealth hell, the glorious raised. \\
Be. Then think your fortune's await tormenteth, praised: \\
}

\setlength{\leftskip}{0pt}

The rhymes are generated well. The poems themselves do not seem to be worse due to the generation method, but since they were mostly nonsense to begin with, it's hard to tell if there's improvement.

\section{Model Selection}
We didn't do anything too sophisticated to do model selection; we trained our HMMs with different numbers of hidden states, and then examined the poems and chose the number of hidden states which made the poems we thought were qualitatively best. Due to the time required to do unsupervised HMM training with large numbers of hidden states, when using all of the poems, we limited our search to numbers of hidden states below 17. 
\par In the end, using hidden states between 8 and 17 didn't really seem to change much. The poems were all gibberish. They could somewhat follow proper grammar, but meaning-wise, there was nothing. Meter seemed to be hit or miss. Due to the very random nature of the resulting poems, though, if the poem generation algorithm was run enough times, surely a great, creative poem could pop out.
\par In order to train on a large number of hidden states, we also used the preprocessing method described earlier to choose just a few poems (3-6 usually) and train on them with numbers of hidden states on the order of the number of unique tokens. One example is where we trained 3 poems with a total of 176 unique tokens using 176 hidden states. The results definitely made more sense and had better meter. However, the cost was that the generated poems had more phrases taken directly from the poems they were trained on. With the number of hidden states equal to the number of poems, however, this was not too much of a problem, as the size of the directly taken phrases seemed to cap at about 4 words. Most of the in the generated poem were still unique to the poem, though, and if phrases were taken, the most common size by far was 2 word phrases.
\par So, in the end, we mostly experimented on two extremes. One where there was a lot of data and few hidden states, and another where there was not much data and many hidden states. Time constraints would not allow us to test out the case of a lot of data and many hidden states. For these two extremes that we were able to examine though, the pros and cons can clearly be seen. If we want more "creativity," we should go for more data and less states (8-12 seemed fine). If we want to be more Shakespeare-like, we should go for more hidden states. 

[TODO: Final choice of model?]

\section{Visualization}
\begin{tabular}{|c|c|c|c|}
\hline 
State & Noun & Verb & Article \\ 
\hline 
0 & 0.53 & 0.30 & 0.00 \\ 
\hline 
1 & 0.12 & 0.07 & 0.00 \\ 
\hline 
2 & 0.48 & 0.36 & 0.10 \\ 
\hline 
3 & 0.23 & 0.07 & 0.19 \\ 
\hline 
4 & 0.51 & 0.31 & 0.00 \\ 
\hline 
5 & 0.10 & 0.05 & 0.00 \\ 
\hline 
6 & 0.17 & 0.13 & 0.00 \\ 
\hline 
7 & 0.28 & 0.14 & 0.00 \\ 
\hline 
\end{tabular} 

We trained essentially two kinds of HMMs: Part of speech modeling HMM and word modeling HMM. For the former, we trained with numbers of hidden states close to the number of parts of speech (8) in English. For the latter, we used large numbers of hidden states, equal to some fraction of the number of unique words present in the poems we used to train the HMM.  For the 8 state HMM, we expected to see some states have high probability of emitting nouns, others verbs and particles. However, as demonstrated by the chart above, we saw multiple states have higher probabilities of emitting all three part of speech types compared to other states. This may be the result of the imperfect methods we use to visualize nouns and verbs however. We determined that there is no good way of determining if a single word, without any context, is a noun or a verb. The method we use is to check words against a large database of already tagged words (specifically nltk's wordnet corpus). However, the corpus we use only has 60k nouns and probably does not include a lot of the older terms used in these poems. Interestingly, state 3 in the chart has a much higher probability of emitting articles compared to other states, and lower probability of emitting nouns and verbs compared to other states. This may indicate that state 3 represents particles. 

For the word modeling HMM, many of the states had 0 or 100 percent emission probability of a noun, verb, or article. This makes sense if each state represents a particular word. 

\section{Conclusion}
\par We trained a variety of hidden Markov models to generate Shakespearean sonnets. Constraints of the sonnet form we could achieve were limiting each line to 10 syllables, and introducing end rhymes. We were not able to replicate the meter, or create semantically meaningful poems. As stated in the overview, it seems that first-order hidden Markov models are too weak to model the complicated interactions that make up a sonnet (without even considering the difficulty of making poems with meaning). Possible directions to improve our model could be more detailed search over the space of hidden states and tokenizations to find better HMMs, research into clever tactics to allow HMMs to capture this structure, and switching to more sophisticated models (such as more complicated graphical models) to try to better capture the sonnet structure.


\end{document}
